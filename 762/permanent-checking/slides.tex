\documentclass{beamer}
\usepackage{listings}
\usepackage{enumerate}
\usepackage{amsmath}
\usepackage{amssymb}
\usepackage{tikz}
\usepackage{graphicx}
\title[A portable compiler-integrated approach to permanent checking]
{A portable compiler-integrated approach to permanent checking}
\date{October 31, 2012}

\author[Nic Volanschi]{Nic Volanschi}
\begin{document}

\AtBeginSection[]
{
    \begin{frame}{Table of Contents}
        \tableofcontents[currentsection]
    \end{frame}
}

\begin{frame}
\nocite{NV}
\nocite{Unparsed}
\nocite{Condate}
\titlepage
\begin{center}
Presented by Ismail Badawi
\end{center}
\end{frame}

\begin{frame}{Table of Contents}
\tableofcontents
\end{frame}

\section{Overview}
\begin{frame}{What are we talking about?}
\begin{itemize}
\item A compiler integrated approach to permanent checking, published in ASE 08
\item (companion conference paper: Condate (PPDP 06))
\item Program checking tools are mature and useful
\begin{itemize}
\item Static analyzers, model checkers, formal verification tools...
\end{itemize}
\item But they're not really widely used. Why not?
\begin{itemize}
\item They can often be too slow for everyday use
\item Some are quite difficult to use
\item They're decoupled from the rest of the development process
\end{itemize}
\item Premise: let's integrate program checking into the compiler
\end{itemize}
\end{frame}

\begin{frame}{Okay, but...}
\begin{itemize}
\item To make this viable, checking shouldn't take too long relative to
the build time
\begin{itemize}
\item e.g. full-blown model checkers take a long time to run; 
programmers aren't going to want to do this on every compile
\end{itemize}
\item Tradeoff: power and precision vs. speed, usability
\item Here we focus on the latter
\item It would also be nice if the approach was as language agnostic as
possible, to make it easy to integrate it into many different compilers
for many different languages
\end{itemize}
\end{frame}

\begin{frame}{Interesting ideas}
\begin{itemize}
\item Unparsed patterns: easy to implement language independent 
pattern-matching technique
\item Program properties expressed as reachability queries over the
control-flow graph, together with data flow constraints
\item Proof of concept; entire thing implemented as a 1000-line patch
to {\tt gcc}, and successfully used to check properties in the Linux kernel
\end{itemize}
\end{frame}

\section{Unparsed patterns}
\begin{frame}[fragile]{Pattern matching}
\begin{itemize}
\item Generally useful in program manipulation tools
\item Typically reduced to tree matching
\item \begin{verbatim}
for(i = 0; i < 100; ++i)
    a[i] = 0;
\end{verbatim}
\begin{itemize}
\item Tree pattern:
\begin{verbatim}
for_stmt(assign_expr(X, int_literal(0)),
         less_expr(X, N),
         preincr_expr(X),
         expr_stmt(assign_expr(array_expr(Y, X), 
                               int_literal(0))))
\end{verbatim}
\item Concrete syntax pattern
\begin{verbatim}
for(%x = 0; %x < %n; ++%x)
    %y[%x] = 0;
\end{verbatim}
\end{itemize}
\end{itemize}
\end{frame}

\begin{frame}{Problems}
\begin{itemize}
\item Tree patterns
\begin{itemize}
\item Easy to parse
\item Requires pattern writer to be familiar with AST structure / notation
\item Verbose, not very readable
\end{itemize}
\item Concrete syntax patterns
\begin{itemize}
\item Much easier to read and write
\item Difficult to parse; essentially need to modify a parser of the
subject language to allow meta-variables, and parsing of arbitrary program
fragments
\end{itemize}
\end{itemize}
\end{frame}

\begin{frame}{Unparsed pattern}
\begin{itemize}
\item A form of concrete syntax pattern; literal program fragments together
with meta-variables
\item Meta-variables: {\tt \%x}, where {\tt x} is a single character, stands
for an arbitrary subtree of an AST
\item e.g. {\tt \%L = \%L->next} matches {\tt l = l->next}, {\tt buf[i] = buf[i]->next}, but not {\tt l = buf[0]->next}
\item (allow {\tt \%\_} for matching trees we don't care about)
\item Key idea: patterns like this can be matched against an AST efficiently without ever parsing them
\end{itemize}
\end{frame}

\begin{frame}{Lazy unparsing}
\begin{itemize}
\item The matching algorithm is based on \emph{lazy unparsing} of an AST.
\item Instead of unparsing an AST to a string, unparse just one level,
returning a list of strings and subtrees
\item e.g. unparsing {\tt a[0] * b[1]} might give
{\tt [AST('a[0]'), '*', AST('b[1]')]}
\item Not difficult to modify an existing pretty-printer to do this; just
add subtrees to a list instead of recursing on them
\end{itemize}
\end{frame}

\begin{frame}{Matching algorithm}
\begin{itemize}
\item Input: an unparsed pattern $p$ (as a string), and an AST $t$
\item Output: a substitution $\sigma$ mapping meta-variables to subtrees of $t$ if successful (or an error)
\item Expressed a set of rewrite rules over triples $\langle s, p, \sigma \rangle$, where $s$ is the unparse stack, $p$ is the pattern stream, and $\sigma$ is the substitution
\item Initial state: $\langle [t], p, \{\} \rangle$. Final state: $\langle [], [], \sigma \rangle$ if successful, or a state in which no rule applies
\end{itemize}
\end{frame}

\begin{frame}{Three basic operations}
\begin{enumerate}
\item \emph{Bind}; bind the meta-variable at the front of the stream to
the tree on top of the stack.
\item \emph{Unparse}; pop the tree at the top of the stack, and push its
unparsed list
\item \emph{Lookahead}; look at the second element on the stack.
\begin{itemize}
\item $lookahead([], []) = true$
\item $lookahead([e|s], k + p) = (first\_token(e) = k)$
\item $lookahead([t|s], "\%x" + p) = true$
\item otherwise $false$
\end{itemize}
\end{enumerate}
\end{frame}

\begin{frame}{Rewrite rules}
\begin{enumerate}
\pause
\item $\langle [k|s], k + p, \sigma \rangle \longrightarrow \langle s, p, \sigma \rangle$ \\
if top of the stack is a string $k$, and the pattern starts with $k$,
consume both
\pause
\item $\langle [t|s], k + p, \sigma \rangle \longrightarrow \langle unparsed(t) + s, k + p, \sigma \rangle $ \\
if top of the stack is a tree $t$, and the pattern starts with a string (as
opposed to a meta-variable), unparse
\pause
\item $\langle [t|s], "\%x" + p, \sigma \rangle \longrightarrow \langle unparsed(t) + s, "\%x" + p, \sigma \rangle $ \\ $ \,\,\,\, \text{if } \neg lookahead(s, p)$ \\
if top of the stack is a tree $t$, and the pattern starts with a meta-variable, but binding would cause lookahead to fail, unparse
\end{enumerate}
\end{frame}

\begin{frame}{Rewrite rules (cont.)}
\begin{enumerate}
\setcounter{enumi}{3}
\pause
\item $\langle [t|s], "\%x" + p, \sigma \rangle \longrightarrow \langle s, p, \sigma \cup \{x \leftarrow t \} \rangle$ \\ $ \,\,\,\, \text{if } x \not \in \sigma \wedge lookahead(s, p)$ \\
if top of the stack is a tree $t$, and the pattern starts with a meta-variable which is not already bound, and binding is safe, then bind
\pause
\item $\langle [t|s], "\%x" + p, \sigma \rangle \longrightarrow \langle (s, p, \sigma \rangle$ \\ $ \,\,\,\, \text{if } x \in \sigma \wedge \sigma(x) = t \wedge lookahead(s, p)$ \\
if top of the stack is a tree $t$, and the pattern starts with a meta-variable which is bound but to a tree equivalent to $t$, and binding is safe, consume both
\end{enumerate}
\end{frame}

\begin{frame}{Example match}
Match {\tt repmat(0, \%x)} against the tree for {\tt repmat(0, 4, 4)}. 
\begin{tiny}
\begin{tabular}{| l | l | l | l |}
\hline
Rule & Stack & Pattern & $\sigma$ \\ \hline
& $[AST({\tt repmat(0, 4, 4)})]$ & {\tt repmat(0, \%x)} & $\{\}$ \\ \hline
2 & $[AST({\tt repmat}), {\tt '('}, AST({\tt 0, 4, 4}), {\tt ')'}]$ & {\tt repmat(0, \%x)} & $\{\}$ \\ \hline
2 & $[{\tt 'repmat'}, {\tt '('}, AST({\tt 0, 4, 4}), {\tt ')'}]$ & {\tt repmat(0, \%x)} & $\{\}$ \\ \hline
1 & $[{\tt '('}, AST({\tt 0, 4, 4}), {\tt ')'}]$ & {\tt (0, \%x)} & $\{\}$ \\ \hline
1 & $[AST({\tt 0, 4, 4}), {\tt ')'}]$ & {\tt 0, \%x)} & $\{\}$ \\ \hline
2 & $[AST({\tt 0}), {\tt ','}, AST({\tt 4,4}), {\tt ')'}]$ & {\tt 0, \%x)} & $\{\}$ \\ \hline
2 & $[{\tt '0'}, {\tt ','}, AST({\tt 4,4}), {\tt ')'}]$ & {\tt 0, \%x)} & $\{\}$ \\ \hline
1 & $[{\tt ','}, AST({\tt 4,4}), {\tt ')'}]$ & {, \%x)} & $\{\}$ \\ \hline
1 & $[AST({\tt 4,4}), {\tt ')'}]$ & {\%x)} & $\{\}$ \\ \hline
4 & $[{\tt ')'}]$ & {\tt )} & $\{x \leftarrow AST({\tt 4,4})\}$ \\ \hline
1 & $[]$ & & $\{x \leftarrow AST({\tt 4,4})\}$ \\ \hline
\end{tabular}
\end{tiny}
\end{frame}

\begin{frame}{Assessment}
Unparsed patterns are nice because
\begin{itemize}
\item They're trivial to write for any programmer
\item The implementation is relatively simple compared to techniques
based on parsing and tree matching
\item Aside from the unparsing (which isn't very much work), the technique
is completely language independent
\end{itemize}
\end{frame}

\begin{frame}{But wait}
The algorithm we've seen matches a pattern against an entire tree; it
doesn't find matches.

Find matches by just iterating over statements and applying the matching
algorithm. It turns out this is fine because the implementation runs
on three-address code (more on this later).
\end{frame}

\section{Condate}
\begin{frame}{What's next?}
\begin{itemize}
\item Unparsed patterns allow us to match arbitrary code constructs, but
they're purely syntactic
\item It would be useful to also incorporate control flow, data flow, 
semantics into our properties
\item It is known that many dataflow analyses can be expressed as graph
reachability problems on an exploded program graph, intuitively the product
of a CFG with a value-flow graph (Reps, POPL '95)
\item Idea: many useful properties can be expressed as reachability problems
directly on the CFG, skipping the construction of the above product
\end{itemize}
\end{frame}

\begin{frame}{Constrained reachability queries}
\begin{itemize}
\item Is there a path from a statement $f$ to a statement $t$ avoiding
statements $v$, where $f$, $t$ and $v$ are unparsed patterns?
\item e.g. memory leaks: is there a path from {\tt \%x = malloc(\%\_)} to
the exit node avoiding {\tt free(\%x)}?
\pause
\item e.g. potential null deference: is there a path from {\tt \%x = malloc(\%\_)}
to {\tt *\%x} avoiding {\tt if (\%x != 0)}?
\pause
\item That last one isn't quite right. What's missing?
\end{itemize}
\end{frame}

\begin{frame}{Outcome of tests}
\begin{itemize}
\item from {\tt \%x = malloc(\%\_)} to {\tt *\%x} avoiding {\tt if (\%x != 0)}
\item We're ignoring the outcome of the check! There could still be null
deferences in the {\tt else} block.
\item We'd like to say: from {\tt \%x = malloc(\%\_)} to {\tt *\%x} avoiding \emph{successful} tests {\tt \%x != 0} (or unsuccessful tests {\tt \%x == 0})
\item Condate encodes this as {\tt +"\%x != 0"} and {\tt -"\%x == 0"}
respectively
\end{itemize}
\end{frame}

\begin{frame}{Condate definition}
$$ S \rightarrow {\tt from} \, D \,  [{\tt to} \, D \, [{\tt avoid} \, D]] $$
$$ D \rightarrow E \, | \, E \, {\tt or } \, E $$
$$ E \rightarrow P \, | \, +P \, | \, -P $$
$$ P \rightarrow "(\%V | lit)*" $$
\pause
\begin{itemize}
\item Omitting the {\tt avoid} clause: default to {\tt ""}, i.e. matches
nothing, so pure reachability
\item Omitting the {\tt to} clause: default to {\tt "\%\_"}, i.e. matching
everything; this is the same as just looking for matches of the {\tt from}
clause
\end{itemize}
\end{frame}

\begin{frame}[fragile]{Example Condate checker}
A simple check for use-after-free bugs:
\begin{verbatim}
from "kfree_skb(%X)" or "dev_kfree_skb_any(%X)" or 
     "kfree(%X)"
to "%_ = %X->%_" or "%X->%_ = %_"
avoid "%X = %_"
\end{verbatim}
\end{frame}

\begin{frame}{Regular path expressions}
\begin{itemize}
\item Take a CFG; each node is a program statement that doesn't contain internal control flow
\item For control flow nodes, outgoing edges corresponding to successful tests are labeled with $+$, otherwise $-$
\item If a pattern $p$ matches a node $n$, it matches all edges leaving $n$
\item $+p$ matches $+$ edges that $p$ matches (similarly $-$)
\item Define regexes as usual, with concatenation representing path
concatenation
\item A query like {\tt from f to t avoid v or +vt or -ve} becomes a regex like {\tt f[\textasciicircum v +vt -ve]*t}
\end{itemize}
\end{frame}

\begin{frame}{Matching algorithm}
\begin{itemize}
\item This formulation motivates the development of the algorithm, as it's
more or less a standard automaton for matching regexes
\item Glossing over some technical details here...
\begin{itemize}
\item Suppose metavariables are only introduced in the {\tt from} clause
\item Find all the matches (and corresponding substitutions) for the
{\tt from} clause
\item For each of these, \emph{instantiate} the query by replacing all
metavariables with their bindings, and \emph{then} check the regex
\item Turns out this is faster
\end{itemize}
\end{itemize}
\end{frame}

\begin{frame}[fragile]{Algorithm}
\begin{verbatim}
check(CFG, from, to, avoid)
  substs = {}
  for node t in CFG
    subst = match(t, from)
    if subst is not empty add subst to subst
  for subst in subst
    from', to', avoid' = instantiate(subst, from, to, avoid)
    worklist = []
    for node t in CFG
      if match(t, from') then add(worklist, t)
\end{verbatim}
\end{frame}

\begin{frame}[fragile]{Algorithm (cont.)}
\begin{verbatim}
    while worklist is not empty
      t = worklist[0]
      if visited(t) then continue
      visited(t) = true
      if match(t, to')
        return success
      else if not match(t, avoid')
        for edge e = t -> t'
          if not match(e, avoid')
            add(t', worklist)
\end{verbatim}
\end{frame}

\begin{frame}{Practical considerations}
\begin{itemize}
\item Unparsed pattern matching works by pretty printing the AST, but often
you want to work on some simplified IR
\begin{itemize}
\item The paper presents an implementation in {\tt gcc}, which uses
GIMPLE three-address form
\end{itemize}
\end{itemize}
\end{frame}

\begin{frame}[fragile]{Example}

{\tt from "lock(\%X, \%Y) to "return" avoid "unlock(\%X, \%Y)"}

\begin{columns}
\column{2in}
\begin{verbatim}
lock(step[i+1], NOWAIT);
critical_section(...);
unlock(step[i+1], NOWAIT);
return;
\end{verbatim}
\column{2in}
\begin{verbatim}
T.2 = i + 1;
T.3 = step[T.2];
lock(T.3, 0);
critical_section(...)
T.4 = i + 1;
T.5 = step[T.4];
unlock(T.5, 0);
\end{verbatim}
\end{columns}
\vspace{10mm}
Notice different temporaries are used for the {\tt lock} and {\tt unlock} calls. Pattern doesn't match anymore! How do we deal with this?
\end{frame} 

\begin{frame}{Basic solution: "inlining" temporaries}
\begin{itemize}
\item Assume temporaries are marked as such in the IR
\item Just look up and use the definition of the temporary instead of the
temporary itself
\item Reconstructing the higher-level syntax; e.g. we still bind {\tt \%X}
to {\tt step[i+1]}
\item This can be done directly in the matching algorithm
\end{itemize}
\end{frame}

\begin{frame}{Value numbering}
\begin{itemize}
\item {\tt gcc} uses global value numbering in building up GIMPLE
\item Since {\tt T.2} and {\tt T.4} both refer to {\tt i + 1}, they'd
actually be the same temporary
\item Same for {\tt T.3 = step[T.2]} and {\tt T.5 = step[T.4]}
\item So in real GIMPLE {\tt T.3} and {\tt T.5} are actually the same!
\item So this matches, but with {\tt \%X = T.3} instead of {\tt step[i+1]};
this might be good enough and we can avoid inlining
\item What we had {\tt lock(\%X + 1, \%Y)}?
\end{itemize}
\end{frame}

\section{Evaluation}
\begin{frame}
As mentioned, Condate was implemented as a {\tt gcc} patch; evaluation
was performed on the Linux kernel, to try and reproduce the detection
of bugs reported by another (much more complicated) tool. Evaluated on
three fronts
\begin{enumerate}
\item Expressiveness
\item Precision
\item Scalability
\end{enumerate}
\end{frame}

\begin{frame}{Testbed}
This previous tool was called Metal, and expressed checks as arbitrary
automata mixed with executable C code. It's obviously strictly more
powerful than Condate, but also pretty complicated.

The study on the Linux kernel included 12 different user-defined checks
and found over 500 bugs, which were all manually confirmed by kernel
developers.

Checks like "do not use freed memory", "do not use floating point in the
kernel", "do not deference user pointers"
\end{frame}

\begin{frame}{Expressiveness}
\begin{itemize}
\item Tried to express as many of the checks as possible using Condate
\item Was able to express 9 of them (technically 11 with a feature I
didn't mention -- calling internal compiler checks -- but they didn't
actually implement it)
\end{itemize}
\end{frame}

\begin{frame}[fragile]{Precision}
\begin{itemize}
\item Picked one checker -- checking for null deference.
\item In the old study, it reported 121 bugs in 89 source files
\item Condate found 117 of the 121
\begin{itemize}
\item Two of the missed bugs are because the allocations are hidden
behind macros
\item The others look like this:
\begin{verbatim}
if (!dev) {
  dev = init_etherdev(dev, 0);
}
dev->priv = kmalloc(sizeof(struct awc_private),
    0x02 | 0x01 | 0x04);
\end{verbatim}
{\tt init\_etherdev()} allocates memory only when its first argument is
NULL. Here it obviously does, but can't tell with Condate; can only only
use e.g. {\tt init\_etherdev(0, \%X)}.
\end{itemize}
\end{itemize}
\end{frame}

\begin{frame}{Precision (cont.)}
\begin{itemize}
\item Two false positives, both because of chained assignment in
conditionals, e.g. {\tt if (!(x->y = x->z = malloc(...)))} and later
dereference {\tt x->z}, but {\tt x->y} is the one that was checked against
NULL
\item Four new bugs!
\end{itemize}
\end{frame}

\begin{frame}{Scalability}
\begin{itemize}
\item Checking overhead related to number of checkers used, number of
matches found per pattern, and pattern size
\item Overhead ranged from 10-15\% for a very simple checker (6 patterns)
to 50-80\% for a complex checker (51 patterns)
\item Maximum overhead with all checkers combined: 98\%
\item (This is a prototype; not very optimized)
\end{itemize}
\end{frame}

\begin{frame}{References}
\bibliographystyle{plain}
{\footnotesize
\bibliography{bib}}
\end{frame}

\end{document}
