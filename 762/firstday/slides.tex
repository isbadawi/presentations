\documentclass{beamer}
\title[COMP762 paper proposals]{COMP762 paper proposals}
\date{September 5, 2012}
\author{Ismail Badawi}

\begin{document}
\begin{frame}
\titlepage
\end{frame}



\begin{frame}{My research area}
\begin{itemize}
\item Applications of compiler tools and techniques to software engineering.
\item This includes things like static program analysis, bug detection, 
programmer tools for e.g. automated refactoring, IDEs...
\item My COMP621 project was McLint, a static code analyzer for MATLAB;
my thesis will likely be something related to that
\end{itemize}
\end{frame}

\begin{frame}{Table of Contents}
\begin{enumerate}
\item Remedying the Eval that Men Do (July 2012) \\ Simon Holm Jensen, Peter A. Jonsson, Anders M\o ller \\ Aarhus University, Denmark
\item Cecil: A Sequencing Constraint Language for Automatic Static Analysis
Generation (1999) \\ Kurt M. Olender, Leon J. Osterweil, University of Colorado
\item A Graph-Free Approach to Data-Flow Analysis (2002) \\ Markus Mohnen, RWTH Aachen University, Germany
\item Combined Static and Dynamic Analysis (2005) \\ Cyrille Artho, ETH Zurich, Switzerland \\ Armin Biere, Johannes Kepler University, Austria
\end{enumerate}
\end{frame}

\begin{frame}{Remedying the Eval that Men Do}
\begin{itemize}
\item Traditional wisdom: "{\tt eval} is evil. Avoid it."
\item Last year, Richards et al. published a study analyzing real-world 
usage of {\tt eval} in JavaScript
\item "We have recorded the behavior of 337 MB of strings given as
arguments to 550,358 calls to the {\tt eval} function exercised in over
10,000 web sites."
\item "Although {\tt eval} is pervasive, we can expect that relatively few
web sites (around 25\%) use {\tt eval} in ways that are truly challenging to
reason about with static analysis."
\item Common example: {\tt eval('object.' + property)} instead of {\tt object[property]} 
\item Idea: "eliminate {\tt eval} calls soundly and automatically by
incorporating refactoring into the fixpoint computation of a dataflow
analyzer"
\end{itemize}
\end{frame}

\begin{frame}{Cecil: A Sequencing Constraint Language for Automatic Static Analysis Generation}
\begin{itemize}
\item Static analysis tools are often limited; typically some small number
of builtin analyses, maybe a plugin mechanism to write more
\item In many cases, the things we want to check are simple enough.
\item Idea: introduce a language that can be used for specifying certain
kinds of constraints, and mechanically generate dataflow analyses to check
them.
\item This paper motivates and introduces the Cecil language and shows in
the abstract how to go about generating analyses
\item Another paper (Cesar: A Static Sequencing Constraint Analyzer) deals
with more pragmatic considerations of how do this with real programs.
\end{itemize}
\end{frame}

\begin{frame}{Motivating example}
\begin{itemize}
\item Say you're working with Fortran, and you define three kinds of events:
$d$, variable definition, $r$, variable reference, and $u$, variable
undefinition
\item Then you can characterize certain kinds of errors as just sequences
of these events: $\langle d, d \rangle$ is an unused variable or redundant
definition, $\langle u, r\rangle$ is referencing an undefined variable.
\item Generalize this; let users specify their own events, and let
constraints be essentially regular expressions over the set of events
\item There's an algorithm for taking a constraint like this and generating
an analysis to check it against a given flowgraph. 
\end{itemize}
\end{frame}

\begin{frame}{A Graph-Free Approach to Data-Flow Analysis}
\begin{itemize}
\item The classical way to do dataflow analysis uses a control flow graph;
start with an initial approximation, and propagate dataflow facts along
control flow edges until you reach a fixed point.
\item This paper presents an alternative algorithm which is 
"execution-based"; rather than model control flow with a CFG, execute the
actual instructions of the program on abstract values.
\item Advantages of this
    \begin{itemize}
    \item Lower memory usage since you don't need a CFG
    \item Better cache locality since you can get away with using only arrays
    \item No need to model the semantics of the program (which typically
    implies some pre or post processing) since you can just execute the program itself
\end{itemize}
\item An implementation targeting the JVM is presented.
\end{itemize}
\end{frame}

\begin{frame}{Combined Static and Dynamic Analysis}
\begin{itemize}
\item Dynamic analysis, as opposed to static analysis, refers to analyzing
the behavior of actual executions of a program on various inputs.
\item This paper builds on the previous one. Idea: with this execution-based
approach to static analysis, static and dynamic analyses are similar enough
to be combined.
\item Try to "abstract out the environment", and end up with generic
analyses that can be run in either a static or dynamic context.
\item This can lead to novel things like running an analysis statically to
guide the selection of test cases, then running it dynamically against
those test cases to verify behavior.
\item An implementation targeting the JVM is presented.
\end{itemize}
\end{frame}

\end{document}
